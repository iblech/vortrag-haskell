% Kompilieren mit: TEXINPUTS=minted/source: pdflatex -shell-escape %
\documentclass[12pt,compress,ngerman,utf8,t]{beamer}
\usepackage[ngerman]{babel}
\usepackage{ragged2e}
\usepackage{comment}
\usepackage{minted}
\usepackage{wasysym}
\usepackage{tikz}
\usetikzlibrary{calc}
\usepackage[all]{xy}
\usepackage[protrusion=true,expansion=false]{microtype}

\DeclareSymbolFont{extraup}{U}{zavm}{m}{n}
\DeclareMathSymbol{\varheart}{\mathalpha}{extraup}{86}
\DeclareMathSymbol{\vardiamond}{\mathalpha}{extraup}{87}

\DeclareUnicodeCharacter{2237}{$\dblcolon$}
\DeclareUnicodeCharacter{21D2}{$\Rightarrow$}
\DeclareUnicodeCharacter{2192}{$\rightarrow$}

\title[Effektsysteme]{\smiley{} Effektsysteme \smiley}
\author[Augsburger Curry Club]{
  Ingo Blechschmidt \\[0.1em] \scriptsize\texttt{<iblech@speicherleck.de>}}
\date[2015-11-05]{\vspace*{-1.5em}\scriptsize Augsburger Curry-Club \\ 5.
November 2015}

\usetheme{Warsaw}

\useinnertheme{rectangles}

\usecolortheme{seahorse}
\definecolor{mypurple}{RGB}{150,0,255}
\setbeamercolor{structure}{fg=mypurple}

\usefonttheme{serif}
\usepackage[T1]{fontenc}
\usepackage{libertine}

\definecolor{darkred}{RGB}{220,0,0}
\newcommand{\hcancel}[5]{%
    \tikz[baseline=(tocancel.base)]{
        \node[inner sep=0pt,outer sep=0pt] (tocancel) {#1};
        \draw[darkred, line width=1mm] ($(tocancel.south west)+(#2,#3)$) -- ($(tocancel.north east)+(#4,#5)$);
    }%
}%

\newcommand{\slogan}[1]{%
  \begin{center}%
    \setlength{\fboxrule}{2pt}%
    \setlength{\fboxsep}{-3pt}%
    {\usebeamercolor[fg]{item}\fbox{\usebeamercolor[fg]{normal
    text}\parbox{0.9\textwidth}{\begin{center}#1\end{center}}}}%
  \end{center}%
}

\renewcommand{\C}{\mathcal{C}}
\newcommand{\D}{\mathcal{D}}
\newcommand{\id}{\mathrm{id}}
\newcommand{\Id}{\mathrm{Id}}
\newcommand{\Hask}{\mathrm{Hask}}

\setbeamertemplate{navigation symbols}{}
\setbeamertemplate{headline}{}

\setbeamertemplate{title page}[default][colsep=-1bp,rounded=false,shadow=false]
\setbeamertemplate{frametitle}[default][colsep=-2bp,rounded=false,shadow=false,center]

\newcommand*\oldmacro{}%
\let\oldmacro\insertshorttitle%
\renewcommand*\insertshorttitle{%
  \oldmacro\hfill\insertframenumber\,/\,\inserttotalframenumber\hfill}

\newcommand{\hil}[1]{{\usebeamercolor[fg]{item}{\textbf{#1}}}}
\setbeamertemplate{frametitle}{%
  \vskip1em%
  \leavevmode%
  \begin{beamercolorbox}[dp=1ex,center]{}%
      \usebeamercolor[fg]{item}{\textbf{\textsf{\Large \insertframetitle}}}
  \end{beamercolorbox}%
}

\setbeamertemplate{footline}{%
  \leavevmode%
  \hfill%
  \begin{beamercolorbox}[ht=2.25ex,dp=1ex,right]{}%
    \usebeamerfont{date in head/foot}
    \insertframenumber\,/\,\inserttotalframenumber\hspace*{1ex}
  \end{beamercolorbox}%
  \vskip0pt%
}

\newcommand{\backupstart}{
  \newcounter{framenumberpreappendix}
  \setcounter{framenumberpreappendix}{\value{framenumber}}
}
\newcommand{\backupend}{
  \addtocounter{framenumberpreappendix}{-\value{framenumber}}
  \addtocounter{framenumber}{\value{framenumberpreappendix}}
}

\setbeameroption{show notes}
\setbeamertemplate{note page}[plain]

\begin{document}

\frame{\titlepage}

\frame{\tableofcontents}

\section{Spezifikation von Instruktionen}
\begin{frame}[fragile]\frametitle{Gewünschte Instruktionen}
  \begin{minted}{haskell}
type St = ...
data StateI :: * -> * where
    Get :: StateI St
    Put :: St -> StateI ()
  \end{minted}

  \vfill
  \slogan{Freie Funktoren und freie Monaden liefern ein allgemeines
  Konstruktionsrezept für Monaden mit gewünschter operationeller Semantik.}
\end{frame}

\begin{frame}[fragile]\frametitle{Gewünschte Instruktionen}
  \begin{minted}{haskell}
type St = ...
data StateI :: * -> * where
    Get :: StateI St
    Put :: St -> StateI ()

type Env = ...
data ReaderI :: * -> * where
    Ask :: ReaderI Env

type Log = ...
data WriterI :: * -> * where
    Tell :: Log -> WriterI ()
  \end{minted}
\end{frame}

\begin{frame}[fragile]\frametitle{Gewünschte Instruktionen}
  \begin{minted}{haskell}
type Err = ...
data ErrorI :: * -> * where
    Throw :: Err -> ErrorI a

data IOI :: * -> * where
    PutStrLn :: String -> IOI ()
    GetLine  :: IOI String
    Exit     :: IOI a
  \end{minted}
\end{frame}


\section{Freie Funktoren}
\begin{frame}[fragile]\frametitle{Freie Funktoren}
  Wir können aus einem Typkonstruktor \verb|t :: * -> *| auf unspektakulärste
  Art und Weise einen Funktor machen:

  \begin{minted}{haskell}
class Functor f where
    fmap :: (a -> b) -> (f a -> f b)

data FreeF t a = MkFreeT (exists r. (r -> a, t r))
data FreeF t a = forall r. MkFreeT (r -> a) (t r)
-- MkFreeT :: (r -> a) -> t r -> FreeF t a

liftF :: t a -> FreeF t a
liftF x = MkFreeT id x

instance Functor (FreeF t) where
    fmap phi (MkFreeT h x) = MkFreeT (phi . h) x
  \end{minted}
\end{frame}

\note{\justifying
  Das Paar bestehend aus dem \emph{freien Funktor~\texttt{FreeF t} über~\texttt{t}}
  und der Funktion~\texttt{liftF} erfüllt folgende \emph{universelle
  Eigenschaft}:
  \medskip

  Ist~\texttt{G} irgendein Funktor und~\texttt{phi :: t a -> G a} irgendeine
  Funktion, so existiert genau eine Funktion~\texttt{phi' :: FreeF t a -> G a}
  mit~\texttt{phi' . liftF = phi}.
  \[ \xymatrix{
    \texttt{t} \ar@{=>}[rr]^{\texttt{phi}} \ar@{=>}[rd]_{\texttt{liftF}} && G \\
    & \texttt{FreeF t} \ar@{==>}[ru]_{\texttt{phi'}}
  } \]
}

\begin{frame}[fragile]\frametitle{Freie Funktoren}
  \begin{minted}{haskell}
class Functor f where
    fmap :: (a -> b) -> (f a -> f b)

data FreeF t a = forall r. MkFreeT (r -> a) (t r)

data StateI :: * -> * where
    Get :: StateI St
    Put :: St -> StateI ()

-- FreeF StateI ist isomorph zu:
data StateF a = Get (St -> a) | Put St a
  \end{minted}
\end{frame}


\section{Freie Monaden}
\begin{frame}[fragile]\frametitle{Freie Monaden}
  Wir können aus einem Funktor \verb|f :: * -> *| auf unspektakulärste
  Art und Weise eine Monade machen:
  \begin{minted}{haskell}
data FreeM f a = Pure a | Roll (f (FreeM f a))

instance (Functor f) => Monad (FreeM f) where
    return x = Pure x

    Pure x >>= k = k x
    Roll u >>= k = Roll (fmap (>>= k) u)
  \end{minted}
\end{frame}

\begin{frame}[fragile]\frametitle{Zusammengesetzt}
  \begin{minted}{haskell}
data FreeF t a = forall r. MkFreeF (r -> a) (t r)
data FreeM f a = Pure a | Roll (f (FreeM f a))

-- Also ist `FreeM (FreeF t) a` isomorph zu:
data Prog t a =
    Pure a | forall r. Step (t r) (r -> Prog t a)

data StateI :: * -> * where
    Get :: StateI St
    Put :: St -> StateI ()
-- `Prog StateI a` ist isomorph zu:
data StateProg a
    = Pure a
    | Get    (St -> StateProg a)
    | Put St (StateProg a)
  \end{minted}
\end{frame}

\begin{frame}[fragile]\frametitle{Operationelle Semantik}
  \begin{minted}{haskell}
data StateProg a
    = Pure a
    | Get    (St -> StateProg a)
    | Put St (StateProg a)

interpret :: StateProg a -> (St -> (a,St))
interpret (Pure x)     st = (x, st)
interpret (Get  k)     st = interpret (k st) st
interpret (Put  st' u) st = interpret u      st'
  \end{minted}
\end{frame}

\end{document}

* Trick/Rezept, um aus Instruktionsspezifikationen Monaden zu erhalten.
  Diese reflektieren aber noch nicht die möglicherweise geforderten Gesetze
  (wie "put x >> put x' == put x'"), sondern sind ein rein syntaktischer
  Ansatz. Zum Leben erweckt wird diese Konstruktion durch Angabe eines
  Interpreters.

* Man kann sehr leicht mehrere Instruktionsspezifikationen zu einer
  kombinieren. Daraus erhält man dann eine Monade, in der man die Befehle der
  Teilspezifikationen verwenden kann. Es ist auch möglich, die Interpreter der
  Teilspezifikationen zu kombinieren.

* Laufzeiteffizienzprobleme

* Olegs Eff

* Kan-Erweiterung
