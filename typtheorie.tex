\documentclass[12pt,compress,ngerman,utf8,t]{beamer}
\usepackage{etex}
\usepackage[ngerman]{babel}
\usepackage{ragged2e}
\usepackage{tabto}
\usepackage{wasysym}
\usepackage{booktabs}
\usepackage{mathtools}
\usepackage{tikz}
\usetikzlibrary{calc}
\usepackage[all]{xy}
\usepackage[protrusion=true,expansion=false]{microtype}

\DeclareSymbolFont{extraup}{U}{zavm}{m}{n}
\DeclareMathSymbol{\varheart}{\mathalpha}{extraup}{86}
\DeclareMathSymbol{\vardiamond}{\mathalpha}{extraup}{87}

\DeclareUnicodeCharacter{2237}{$\dblcolon$}
\DeclareUnicodeCharacter{21D2}{$\Rightarrow$}
\DeclareUnicodeCharacter{2192}{$\rightarrow$}

\title[Was sind und was sollen die Typen?]{\smiley{} Was sind und was sollen die Typen? \smiley}
\author[Augsburger Curry Club]{\textcolor{white}{Ingo Blechschmidt \\ Curry Club Augsburg}}
\date[2016-07-14]{\textcolor{white}{14. April 2016}}

\usetheme{Warsaw}

\useinnertheme{rectangles}

\usecolortheme{seahorse}
\definecolor{mypurple}{RGB}{150,0,255}
\setbeamercolor{structure}{fg=mypurple}
\definecolor{myred}{RGB}{150,0,0}
\setbeamercolor*{title}{bg=myred,fg=white}
\setbeamercolor*{titlelike}{bg=myred,fg=white}

\usefonttheme{serif}
\usepackage[T1]{fontenc}
\usepackage{libertine}

\newcommand{\slogan}[1]{%
  \begin{center}%
    \setlength{\fboxrule}{2pt}%
    \setlength{\fboxsep}{8pt}%
    {\usebeamercolor[fg]{item}\fbox{\usebeamercolor[fg]{normal text}\parbox{0.33\textwidth}{#1}}}%
  \end{center}%
}

\definecolor{darkred}{RGB}{220,0,0}
\newcommand{\hcancel}[5]{%
    \tikz[baseline=(tocancel.base)]{
        \node[inner sep=0pt,outer sep=0pt] (tocancel) {#1};
        \draw[darkred, line width=1mm] ($(tocancel.south west)+(#2,#3)$) -- ($(tocancel.north east)+(#4,#5)$);
    }%
}%

\renewcommand{\C}{\mathcal{C}}
\newcommand{\D}{\mathcal{D}}
\newcommand{\id}{\mathrm{id}}
\newcommand{\Id}{\mathrm{Id}}
\newcommand{\Hask}{\mathrm{Hask}}
\newcommand{\defeq}{\vcentcolon=}
\newcommand{\base}{\mathsf{base}}
\newcommand{\lloop}{\mathsf{loop}}
\newcommand{\surf}{\mathsf{surf}}
\newcommand{\merid}{\mathsf{merid}}
\renewcommand{\U}{\mathcal{U}}
\newcommand{\ap}{\mathsf{ap}}
\newcommand{\IsEquiv}{\mathsf{IsEquiv}}
\newcommand{\fib}{\mathsf{fib}}
\newcommand{\UIP}{\mathsf{UIP}}
\newcommand{\ZZ}{\mathbb{Z}}
\newcommand{\inv}{\mathsf{inv}}
\newcommand{\defeqv}{\vcentcolon\equiv}
\newcommand{\IsContr}{\mathsf{IsContr}}
\newcommand{\refl}{\mathsf{refl}}
\newcommand{\IsMereProp}{\mathsf{IsMereProp}}
\newcommand{\NN}{\mathbb{N}}
\newcommand{\ct}{%
  \mathchoice{\mathbin{\raisebox{0.5ex}{$\displaystyle\centerdot$}}}%
             {\mathbin{\raisebox{0.5ex}{$\centerdot$}}}%
             {\mathbin{\raisebox{0.25ex}{$\scriptstyle\,\centerdot\,$}}}%
             {\mathbin{\raisebox{0.1ex}{$\scriptscriptstyle\,\centerdot\,$}}}
}

\setbeamertemplate{navigation symbols}{}
\setbeamertemplate{headline}{}

\setbeamertemplate{title page}[default][colsep=-1bp,rounded=false,shadow=false]
\setbeamertemplate{frametitle}[default][colsep=-2bp,rounded=false,shadow=false,center]

\newcommand*\oldmacro{}%
\let\oldmacro\insertshorttitle%
\renewcommand*\insertshorttitle{%
  \oldmacro\hfill\insertframenumber\,/\,\inserttotalframenumber\hfill}

\newcommand{\hil}[1]{{\usebeamercolor[fg]{item}{\textbf{#1}}}}
\setbeamertemplate{frametitle}{%
  \vskip1em%
  \leavevmode%
  \begin{beamercolorbox}[dp=1ex,center]{}%
      \usebeamercolor[fg]{item}{\textbf{\textsf{\Large \insertframetitle}}}
  \end{beamercolorbox}%
}

\setbeamertemplate{footline}{%
  \leavevmode%
  \hfill%
  \begin{beamercolorbox}[ht=2.25ex,dp=1ex,right]{}%
    \usebeamerfont{date in head/foot}
    \insertframenumber\,/\,\inserttotalframenumber\hspace*{1ex}
  \end{beamercolorbox}%
  \vskip0pt%
}

\newcommand{\backupstart}{
  \newcounter{framenumberpreappendix}
  \setcounter{framenumberpreappendix}{\value{framenumber}}
}
\newcommand{\backupend}{
  \addtocounter{framenumberpreappendix}{-\value{framenumber}}
  \addtocounter{framenumber}{\value{framenumberpreappendix}}
}

\setbeameroption{hide notes}
\setbeamertemplate{note page}[plain]

\newcommand{\imgslide}[1]{{\usebackgroundtemplate{\parbox[c][\paperheight][c]{\paperwidth}{\centering\includegraphics[height=\paperheight]{#1}}}\begin{frame}[plain]\end{frame}}}
\newcommand{\imgslideW}[1]{{\usebackgroundtemplate{\parbox[c][\paperheight][c]{\paperwidth}{\centering\includegraphics[width=\paperwidth]{#1}}}\begin{frame}[plain]\end{frame}}}

\begin{document}

% http://www.unoosa.org/res/timeline/index_html/space-2.jpg
{\usebackgroundtemplate{\includegraphics[height=\paperheight]{images/space}}
\frame{\titlepage}}

{\usebackgroundtemplate{\includegraphics[height=\paperheight]{images/dedekind-titleslide}}
\frame{}}

\frame{\tableofcontents}


\section{Vorgeschichte}

\begin{frame}{Was sind Grundlagen?}
  \begin{itemize}
    \item Grundlagen liefern den logischen Rahmen für Mathe.
    \item Ihre Details spielen oft keine Rolle.
    \item Aber ihre wesentlichen Konzepte schon.
  \end{itemize}

  \bigskip
  \centering
  \includegraphics[scale=0.25]{images/bridge}
  \par
\end{frame}

\imgslideW{images/logicomix-1}
\imgslideW{images/logicomix-2}


\note{
  \begin{itemize}
    \item\justifying Foundations allow us to be maximally precise.
    \item A \emph{proof} as commonly understood is really a shorthand for a
    (never spelled out) fully formal proof.
    \item Unlike informal proofs, the correctness of a formal proof can be
    checked mechanically.
  \end{itemize}

  %\img{0.5}{logicomix}{Logicomix: An Epic Search for Truth}
}
\note{
  \begin{itemize}
    \item\justifying There is no such theorem as ``the sun system is stable if
    and only if the following large cardinal axiom holds''. Results depend only
    very occasionally on special foundational axioms.

    \item Bridges will continue to hold even if a logician discovers an
    inconsistency in Zermelo--Fraenkel set theory.
  \end{itemize}
}

\begin{frame}[plain,c]
  \begin{columns}[c]
    \begin{column}{0.6\textwidth}
      \includegraphics[scale=0.8]{images/hilbert}
    \end{column}
    \begin{column}{0.4\textwidth}
      Aus dem Paradies, das Cantor uns geschaffen, soll uns niemand vertreiben
      können.\medskip

      -- David Hilbert, 1926

      \bigskip
      \bigskip
      \includegraphics[width=\textwidth]{images/ordinal-numbers}
    \end{column}
  \end{columns}
\end{frame}

%\imgslide{images/philosophiae-newton}
\imgslide{images/principia-mathematica}
\imgslide{images/principia-mathematica-1plus1}

{\logo{\includegraphics[scale=0.3]{images/paradox}}
\begin{frame}{Paradoxa}
  \begin{itemize}
    \item Paradox von Richard
    \item Paradox von Curry
    \item Die Russellsche Antinomie
  \end{itemize}

  Was ist all diesen Paradoxa gemein?
  \pause

  \hil{Selbstbezüglichkeit.}
  \medskip

  Lösungsvorschläge: \\
  Mengenlehre, Typtheorie.
\end{frame}}


\section{Mengenlehre}

\begin{frame}{Mengenlehre}
  \slogan{Alles ist eine Menge.}
  \bigskip

  \begin{itemize}
    \item $0 \defeq \emptyset$, \quad
          $1 \defeq \{0\}$, \quad
          $2 \defeq \{0,1\}$, \quad
          $\ldots$
    \item $(x,y) \defeq \{ \{x\}, \{x,y\} \}$ \quad (Kuratowski-Paarung)
    \item $(x,y,z) \defeq (x,(y,z))$
    \item Abbildungen sind Tupel $(X,Y,R)$ mit $R \subseteq X \times Y$ und
    \ldots
    \item \hil{Eingeschränktes Mengenkomprehensionsprinzip.}
  \end{itemize}
\end{frame}

\begin{frame}{Mengenlehre?}
  Mengentheoretische Grundlagen \ldots
  \begin{itemize}
    \item spiegeln nicht die typisierte mathematische Praxis wieder,
    \item respektieren nicht Äquivalenz von Strukturen und
    \item benötigen komplexe Kodierungen von höheren Konzepten --
    zum Nachteil interaktiver Beweisumgebungen.
  \end{itemize}
\end{frame}


\section{Extensionale Typtheorie}

{\logo{\includegraphics[scale=0.3]{images/typsystem}}
\begin{frame}{Extensionale Typtheorie}
  \begin{itemize}
    \item In Typtheorie gibt es \hil{Werte} und \hil{Typen}.
    \item Jeder Wert ist von genau einem Typ.
    \item Es gibt kein globales Gleichheitsprädikat.
  \end{itemize}

  Aus meiner Sicht beschreibt extensionale Typtheorie genau die Vorstellung von
  Mathematikerinnen.
  \medskip

  Wie spezifiert man ein Typsystem?
\end{frame}}


\section{Intensionale Typtheorie}

{\logo{\includegraphics[scale=0.1]{images/torus}}
\begin{frame}{Intensionale Typtheorie}
  \begin{itemize}
    \item Intensionale Typtheorie ist wie extensionale Typtheorie,
    nur ohne das Konzept von Aussagen.
    \item Homotopietyptheorie ist intensionale Typtheorie plus das
    Univalenzaxiom.
  \end{itemize}
\end{frame}}

%\frame[t]{\frametitle{What is homotopy type theory?}
%  \begin{itemize}
%    \item Homotopy type theory is a new foundational theory.
%    \item Basic notions have a homotopy-theoretic flavour.
%    \item One can start doing ``real mathematics'' right away, without complex encodings.
%    \item Initiated by Voevodsky in 2005.
%  \end{itemize}
%
%  %\img{0.2}{authors}{Some participants of the IAS special year}
%}

%\frame<1>[t,label=awesome]{\frametitle{What is homotopy type theory?}
%  Homotopy type theory \ldots
%  \begin{itemize}
%    \item is elegant,
%    \item reflects mathematical practice,
%    \item contains wondrous new concepts,
%    \item ensures that everything respects equivalences,
%    \item simplifies the plumbing of homotopy theory,
%    \item allows for accessible computer formalization.
%  \end{itemize}
%}
%
%\note{
%  \begin{itemize}
%    \item\justifying Homotopy type theory is approximately \emph{intensional Martin-Löf type
%    theory} (existing since the 1970s) plus the new \emph{univalence axiom}.
%    \item After repeatedly experiencing mistakes in his field going unnoticed
%    for several years, Voevodsky wanted to work with proof assistants. He went
%    public in 2009.
%    \item Voevodsky: ``This story got me scared. Starting from 1993 multiple groups of
%    mathematicians studied the [\ldots] paper at seminars and used it in their
%    work and none of them noticed the mistake.
%
%    And it clearly was not an accident. A technical argument by a trusted author, which 
%    is hard to check and looks similar to arguments known to be correct, is
%    hardly ever checked in detail.''
%  \end{itemize}
%}


\frame[t]{\frametitle{Gleichheitstypen}
  Sei~$X$ eine Menge und seien~$x,y \in X$ Elemente.
  \begin{itemize}
    \item Dann ist ``$x=y$'' eine \hil{Aussage}.
  \end{itemize}
  \bigskip

  Sei~$X$ ein Typ in intensionaler Typtheorie und seien~$x,y : X$ Werte.
  \begin{itemize}
    \item Es gibt einen \hil{Gleichheitstyp} $\Id_X(x,y)$ or $(x =_X y)$.
    \item Um ``$x=y$'' nachzuweisen,
    gib einen Wert von~$(x = y)$ an.
    \item Wir haben $\refl_x : (x = x)$.
    \item Gleichheitstypen können Null oder \hil{viele} Werte enthalten!
  \end{itemize}
  Intuition: $(x = y)$ ist der Typ der \hil{Beweise} von ``$x=y$''.

  \pause
  Intuition: $(x = y)$ ist der Typ der \hil{Pfade} $x \leadsto y$.
}

\note{
  \begin{itemize}
    \item\justifying Note that we use logical terminology. A proposition is merely a
    statement, not necessarily a true statement.
    \item In an intensional type theory, propositions are not an extra part of
    the language distinct from values and types.
    \item Instead, \emph{propositions are types}.
    \item To prove a proposition means to exhibit a value of it.
    Such a value can be thought of as a \emph{proof} or \emph{witness}.
    \item We have \emph{proof relevance}.
    \item Types whose values are all equal -- types for which merely knowing that they
    are inhabited is all there is to know -- are called \emph{mere propositions}. See below for
    $\IsMereProp$.
  \end{itemize}
}

\note{
  \justifying
  Examples for more complex propositions (types):
  \begin{itemize}
    \item ``$X$ is a subsingleton'': \tabto{4.9cm}
    $\prod_{x:X} \prod_{y:X} (x=y)$
    \item ``Addition is commutative'': \tabto{4.9cm}
    $\prod_{n:\NN} \prod_{m:\NN} (n+m = m+n)$
    \item ``Every number is even'': \tabto{4.9cm}
    $\prod_{n:\NN} \sum_{m:\NN} (n=2m)$
  \end{itemize}

  By reading ``for all $x:X$'' for ``$\prod_{x:X}$'' and ``there exists $x:X$''
  for~``$\sum_{x:X}$'', these types can be interpreted in a simple logical way.
  But at the same time, they can be read in geometric/homotopy-theoretic terms;
  see below.
  \par
}

\note{
  The type of monoid structures on a type~$X$ is
  \begin{multline*}
    \sum_{\circ:X \times X \to X}
    \sum_{e:X}
    \Biggl(
      \Bigl(\prod_{x:X} (e \circ x = x)\Bigr) \times
      \Bigl(\prod_{x:X} (x \circ e = x)\Bigr) \times \\
      \Bigl(\prod_{x,y,z:X} \bigl((x \circ y) \circ z = x \circ (y \circ z)\bigr)\Bigr)\Biggr).
  \end{multline*}
}

\note{
  \begin{itemize}
    \item\justifying Identity witnesses can be composed: Let $p : (x = y)$ and $q :
    (y = z)$. Then there exists a canonically defined witness~$p \ct q : (x = z)$.

    \item Composition of identity witnesses is associative. The proof of this
    fact is a value of the type
    \[ (p \ct (q \ct r) = (p \ct q) \ct r). \]
  \end{itemize}
}


%\section{Homotopy theory in HoTT}
%
%\subsection{How are types like spaces?}
%
%\frame[t]{\frametitle{How are types like spaces?}
%  \begin{center}\begin{tabular}{ll}
%    \toprule
%    homotopy theory & type theory \\\midrule
%    \hil{space} $X$ & type $X$ \\
%    \hil{point} $x \in X$ & value $x:X$ \\
%    \hil{path} $x \leadsto y$ & value of $(x = y)$ \\
%    \hil{(continuous) map} & value of $X \to Y$ \\
%    \bottomrule
%  \end{tabular}\end{center}
%
%  \begin{itemize}
%    \item A \hil{homotopy} between maps $f, g : X \to Y$ is a value of
%    \[ (f \simeq g) \defeqv \prod_{x:X} (f(x) = g(x)). \]
%    \item A space $X$ is \hil{contractible} iff
%    \[ \IsContr(X) \defeqv \sum_{x:X} \prod_{y:X} (x=y). \]
%  \end{itemize}
%}
%
%%\note{
%%  \begin{center}
%%    \scalebox{0.6}{\input{images/paths.pspdftex}}
%%  \end{center}
%%  \vspace{-1.3em}
%%
%%  \begin{itemize}
%%    \item\justifying This type has values $x, y : X$.
%%    \item The paths~$p$, $q$, and $r$ are values of~$(x = y)$.
%%    \item Since~$p$ and~$q$ are homotopic, we have~$(p = q)$; a witness of this
%%    fact is the value~$H : (p = q)$.
%%    \item Because of the hole, $p$ (and~$q$) are not homotopic to~$r$. So $\neg(p
%%    = r)$; more precisely, the type~$\neg(p=r) \defeqv ((p=r) \to
%%    \boldsymbol{0})$ is inhabited, where~$\boldsymbol{0}$ is the \emph{empty
%%    type}.
%%    \item We have~$(x = y) \simeq \ZZ$, where~$\simeq$ denotes
%%    \emph{equivalence}, to be discussed below.
%%  \end{itemize}
%%}
%
%\note{
%  \begin{itemize}
%    \item\justifying The type $(f \simeq g)$ is the type of homotopies
%    between~$f$ and~$g$. It is read as the type of ``continuous families of
%    paths $f(x) \leadsto g(x)$''.
%
%    \item To understand the definition of contractibility geometrically, one
%    may not read it in logical terms: One may not read it as ``there exists a
%    point~$x$ such that any point~$y$ is connected to~$x$''.
%
%    \item Instead, it should be read as follows: There exists a point~$x$ such
%    that there is a \emph{continuous} way of associating to any point~$y$ a
%    path~$x \leadsto y$. Convince yourself that this not possible for your
%    favorite example of a non-contractible space (for instance, the circle).
%
%    \item A space is \emph{connected} if and only if
%    \[ \sum_{x:X} \prod_{y:X} \|x=y\|_{-1}. \]
%    Here, $\|(x=y)\|_{-1}$ is the \emph{$(-1)$-truncation} of~$(x = y)$; it is a
%    mere proposition. See below.
%  \end{itemize}
%}

\frame[t]{\frametitle{Induktive Definitionen}
%  The \hil{interval} $I$ is generated by
%  \begin{itemize}
%    \item a point $0 : I$ and
%    \item a point $1 : I$ and
%    \item a path $\seg : (0 = 1)$.
%  \end{itemize}
%  Of course, we can show $I \simeq \boldsymbol{1}$.
%  \bigskip
%  \pause

  Der \hil{Kreislinie} $S^1$ wird erzeugt von
  \begin{itemize}
    \item einem Punkt $\base : S^1$ und
    \item einem Pfad $\lloop : (\base = \base)$.
  \end{itemize}
  \bigskip

  Die \hil{Kugeloberfläche} $S^2$ wird erzeugt von
  \begin{itemize}
    \item einem Punkt $\base : S^2$ und
    \item einem 2-Pfad $\surf : (\refl_\base = \refl_\base)$.
  \end{itemize}
  \bigskip

  Der \hil{Torus} $T^2$ wird erzeugt von
  \begin{itemize}
    \item einem Punkt $b : T^2$,
    \item einem Pfad $p : (b = b)$,
    \item einem Pfad $q : (b = b)$ und
    \item einem 2-Pfad $t : (p \ct q = q \ct p)$.
  \end{itemize}
}

\note{
  \begin{itemize}
    \item\justifying Note that a presentation of a type \emph{determines}, but does not
    \emph{explicitly describe} its higher identity types.
    \item Just like the free vector space spanned by set contains not only the
    given elements, but also their linear combinations, the type given by a
    higher inductive definition (or its higher identity types) may contain many
    more values than explicitly listed.
    \item For instance, there is a nontrivial element in $(\refl_{\refl_\base}
    = \refl_{\refl_\base})$, where $\base : S^2$, corresponding to the
    \emph{Hopf fibration}.
    \item More generally, higher-dimensional paths are forced into existence by
    \emph{proofs}. For instance, in~$(\base = \base)$ where $\base : S^1$,
    there are the values~$\lloop \ct (\lloop \ct \lloop)$ and~$(\lloop \ct
    \lloop) \ct \lloop$. They are the same by a witness of type~$(\lloop \ct
    (\lloop \ct \lloop) = (\lloop \ct \lloop) \ct \lloop)$.
    \item Also, different generators may turn out to give rise to the same
    element.
  \end{itemize}
}

\frame[t]{\frametitle{Pfadinduktion}
  \hil{(Based) path induction} sagt aus: Gegeben
  \begin{itemize}
    \item ein Wert~$a$ eines Typs~$A$,
    \item eine Typfamilie~$C : \prod_{x:A} ((a = x) \to \U)$ und
    \item einen Wert~$c : C(a,\refl_a)$,
  \end{itemize}
  dann gibt es eine Funktion
  \[ f : \prod_{x:A} \prod_{p:(a=x)} C(x,p) \]
  mit~$f(a,\refl_a) \equiv c$.
}

\note{
  \begin{itemize}
    \item\justifying In particular, in proving that a proposition depending on a value~$x$
    and an identity witness~$p : (a = x)$ holds for all thoses values and
    witnesses, it suffices to prove it for the special value~$a$ and the
    canonical identity witness~$\refl_a$.

    \item Note that this does not mean that any value of~$(a = x)$ is equal
    to~$\refl_a$! Indeed, this claim is not even well-typed.

    \item The induction principle only makes a statement about the whole \emph{type
    family} of the~$(a = x)$'s with~$x$ varying, not about individual types.

    \item Compare with the classical based path space: In it, any element (any
    path starting at~$a$) is connected to the trivial path at~$a$. But this
    does not mean that any path is homotopic to the trivial path.
  \end{itemize}
}

\note{
  \justifying
  As an example, let's define the path reversal function
  \[ \inv : \prod_{x:A} ((a = x) \longrightarrow (x = a)), \]
  where~$a:A$ is a fixed value, by path induction. For this, we define the type
  family
  \[ C \defeqv ((x:A, p : (a = x)) \mapsto (x = a)) : \prod_{x:A} ((a = x) \to \U) \]
  and note that we have the value
  \[ c \defeqv \refl_a : C(a,\refl_a). \]
  Therefore, by path induction, we obtain a function
  \[ f : \prod_{x:A} \prod_{p:(a=x)} (x=a). \]
  This is~$\inv$.
}

\note{
  \begin{itemize}
  \item\justifying
  Working informally, we would write the construction more concisely:

  ``Let~$p:(a=x)$ be given, we want to construct~$\inv(p) : (x=a)$.
  By path induction, we may assume that~$p \equiv \refl_a$. In this case, we
  define~$\inv(p)$ as~$\refl_a$.''

  \item Here is how we construct the path composition function:

  ``Let~$p:(a=b)$ and~$q:(b=c)$ be given. By path induction, we may assume
  that~$p \equiv \refl_a$. Again by path induction, we may assume that~$q
  \equiv \refl_a$. In this case, we define~$p \ct q$ as~$\refl_a$.''

  \item Here is how to construct the function~$\ap_g : (x = y) \to (g(x) = g(y))$, if~$g$ is
  some given function:

  ``By path induction, it suffices to define~$\ap_g(p)$ when~$p$ is~$\refl_x$.
  In this case, we declare~$\ap_g(p)$ to be~$\refl_{g(x)}$.''
  \end{itemize}
}

\note{
  \justifying
  Path induction does not allow to replace any paths whatsoever by the
  canonical reflexivity witnesses. For instance, the following ``proof'' of
  \[ p \ct q = q \ct p \]
  for all paths~$p$ and~$q$ is bogus:

  ``By path induction, we may assume that~$p$ and~$q$ are~$\refl_a$. In this
  case,~$p \ct q$ and~$q \ct p$ both equal~$\refl_a \ct \refl_a = \refl_a$.''

  Indeed, the claimed statement is not even well-typed:
  \[ \prod_{x,y,z:X} \prod_{p:(x=y)} \prod_{q:(y=z)}
    (p \ct q = q \ct p). \]
  The composition~$q \ct p$ is not defined. Weakening the statement to
  \[ \prod_{x:X} \prod_{p:(x=x)} \prod_{q:(x=x)} (p \ct q = q \ct p) \]
  does not help either, since in this statement there is no free endpoint,
  so path induction does not apply.
}


\end{document}

\subsection{What is the univalence axiom?}

\frame[t]{\frametitle{What is the univalence axiom?}
  An \hil{equivalence} is a function $f : X \to Y$ such that
  \[ \IsEquiv(f) \defeqv \prod_{y:Y} \IsContr(\fib_f(y)). \]

  Types $X$ and $Y$ are \hil{equivalent} iff
  \[ (X \simeq Y) \defeqv \sum_{f : X \to Y} \IsEquiv(f). \]

  The \hil{univalence axiom} states: The canonical function
  \[ (X = Y) \longrightarrow (X \simeq Y) \]
  is an equivalence, for all types $X$ and $Y$.
}

\note{
  \begin{itemize}
    \item\justifying Read in logical terms, a function $f : X \to Y$ is an
    equivalence if and only if for any~$y:Y$, the fiber~$\fib_f(y)$ is a
    singleton, i.\,e.\@ if and only if~$f$ is bijective.
    \item One can prove that~$\IsMereProp(\IsEquiv(f))$.
    \item A value of~$(X \simeq Y)$ is a pair consisting of a function~$f : X
    \to Y$ together with a proof that~$f$ is an equivalence.
  \end{itemize}
}

\note{
  \begin{itemize}
    \item\justifying Let~$X$ and~$Y$ be types, i.\,e.\@ values of the
    universe~$\U$. Then there is the identity type~$(X = Y)$. What does it
    look like?
    \item Without the univalence axiom, this question does not have an answer;
    the special behaviour of~$(X = Y)$ is left unspecified by the remaining rules
    of homotopy type theory.
    \item With the univalence axiom, the question has a clear answer: The
    type~$(X = Y)$ of identity witnesses is equivalent to the type~$(X \simeq
    Y)$ of equivalences.
  \end{itemize}
}

\note{
  \begin{itemize}
    \item\justifying The canonical function~$(X = Y) \to (X \simeq Y)$
    appearing in the univalence axiom is constructed by path induction. It maps
    the canonical reflexivity witness~$\refl_X$ to the trivial
    equivalence~$\id_X : X \to X$ (together with a proof that~$\id_X$ is an
    equivalence).
  \end{itemize}
}

\note{
  \begin{itemize}
    \item\justifying By the univalence axiom, equivalent types \emph{really are} equal.
    \item It implies that isomorphic groups, vector spaces, \ldots\@ are equal.
    \item Thus the widespread practice of \emph{pretending} that isomorphic
    structures are equal is rigorously formalized.
    \item Because any construction has to respect equality, the univalence
    axiom guarantees that \emph{any construction respects equivalence}.
    \item Most nontrivial mathematical results in homotopy type theory require
    the univalence axiom.
  \end{itemize}
}

\note{
  \begin{itemize}
    \item\justifying The univalence axiom implies \emph{function
    extensionality}: The canonically defined function
    \[ (f = g) \longrightarrow \prod_{x:A} (f(x) = g(x)) \]
    is an equivalence, for all functions $f,g : A \to B$.
    \item So homotopic functions are equal.
  \end{itemize}
}

\note{
  \begin{itemize}
    \item\justifying Without the univalence axiom, it is consistent to assume
    \emph{uniqueness of identity proofs}, i.\,e.\@
    \[ \UIP \defeqv \prod_{X:\U} \prod_{x,y:X} \prod_{p,q:(x=y)} (p=q), \]
    thus collapsing the homotopical universe.
    \item Phrased differently, the univalence axiom can not be added to an
    \emph{extensional} type theory (one fulfilling $\UIP$).
  \end{itemize}

  \begin{itemize}
    \item\justifying \emph{No computational interpretation of the univalence axiom is
    known yet.} This prevents us from \emph{running} proofs (as computer
    programs). If this were possible, we could, for instance, simply run a
    proof of the fact that some~$\pi_k(S^n)$ is cyclic (i.\,e.\@ of the
    form~$\ZZ/(m)$) to find out the value of~$m$.
  \end{itemize}
}


\end{document}
