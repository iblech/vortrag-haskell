% Kompilieren mit: TEXINPUTS=minted/source: xelatex -shell-escape %
\documentclass[12pt,compress,ngerman,utf8,t]{beamer}
\usepackage[ngerman]{babel}
\usepackage{minted}
\setminted{linenos}
\usepackage[protrusion=true,expansion=false]{microtype}

\title{Haskell, eine rein funktionale Programmiersprache}
\author{Ingo Blechschmidt \texttt{<iblech@web.de>}}
\date{Augsburger NerdNight am 13. März 2015}

\usetheme{Warsaw}

\useinnertheme{rectangles}

\usecolortheme{seahorse}
\definecolor{mypurple}{RGB}{150,0,255}
\setbeamercolor{structure}{fg=mypurple}

\usefonttheme{serif}
\usepackage{fontspec}
\defaultfontfeatures{Mapping=tex-text}
\setmainfont{Linux Libertine O}

\setbeamertemplate{navigation symbols}{}
\setbeamertemplate{headline}{}

\setbeamertemplate{frametitle}[default][colsep=-2bp,rounded=false,shadow=false,center]

\renewcommand*\insertshorttitle{%
  Erste Augsburger Nerdnight im OpenLab am 13. M\"arz 2015}

\newcommand{\hil}[1]{{\usebeamercolor[fg]{item}{\textbf{#1}}}}

\begin{document}

\frame[plain]{\begin{center}
  \includegraphics[scale=0.35]{images/learn-you-a-haskell-for-great-good.png}
\end{center}}

% Was ist schneller als C++, prägnanter als Perl, regelmäßiger als Python,
% flexibler als Ruby, typisierter als C#, robuster als Java und hat
% absolut nichts mit PHP gemeinsam? Es ist Haskell!

\begin{frame}[fragile]\frametitle{Haskell, eine rein funktionale Sprache}
  \large\vspace{-0.5em}
  \begin{minted}{haskell}
qsort []     = []
qsort (x:xs) =
    qsort kleinere ++ [x] ++ qsort groessere
    where
    kleinere  = [y | y <- xs, y <= x]
    groessere = [y | y <- xs, y > x]
  \end{minted}
  \vfill

  \pause
  Die Fibonaccizahlen: 0, 1, 1, 2, 3, 5, 8, 13, 21, 34, \ldots
  \begin{minted}{haskell}
fibs = 0 : 1 : zipWith (+) fibs (tail fibs)
  \end{minted}

  \pause
  \begin{center}
  \hil{Statisches Typsystem mit Typerschlie\ss ung} \\
  \hil{rein funktional} \textbullet{}
  \hil{nebenl\"aufig} \textbullet{}
  \hil{lazy} \textbullet{}
  \hil{7000\textsuperscript{+} Pakete}
  \end{center}
\end{frame}

% <audreyt>
% Perl: "Easy things are easy, hard things are possible"
% Haskell: "Hard things are easy, the impossible just happened"

\end{document}
