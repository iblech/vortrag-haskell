% Kompilieren mit: TEXINPUTS=minted/source: xelatex -shell-escape %
\documentclass[12pt,compress,ngerman,utf8,t]{beamer}
\usepackage[ngerman]{babel}
\usepackage{comment}
\usepackage{minted}
\setminted{linenos}
\usepackage[protrusion=true,expansion=false]{microtype}

\DeclareSymbolFont{extraup}{U}{zavm}{m}{n}
\DeclareMathSymbol{\varheart}{\mathalpha}{extraup}{86}
\DeclareMathSymbol{\vardiamond}{\mathalpha}{extraup}{87}

\title{Haskell, eine rein funktionale Programmiersprache}
\author{Ingo Blechschmidt \texttt{<iblech@web.de>}}
\date{Augsburger NerdNight am 13. März 2015}

\usetheme{Warsaw}

\useinnertheme{rectangles}

\usecolortheme{seahorse}
\definecolor{mypurple}{RGB}{150,0,255}
\setbeamercolor{structure}{fg=mypurple}

\usefonttheme{serif}
\usepackage{fontspec}
\defaultfontfeatures{Mapping=tex-text}
\setmainfont{Linux Libertine O}

\setbeamertemplate{navigation symbols}{}
\setbeamertemplate{headline}{}

\setbeamertemplate{frametitle}[default][colsep=-2bp,rounded=false,shadow=false,center]

\renewcommand*\insertshorttitle{%
  Erste Augsburger Nerdnight im OpenLab am 13. M\"arz 2015}

\newcommand{\hil}[1]{{\usebeamercolor[fg]{item}{\textbf{#1}}}}

\begin{document}

\frame[plain]{\begin{center}
  \includegraphics[scale=0.35]{images/learn-you-a-haskell-for-great-good.png}
\end{center}}

% Was ist schneller als C++, prägnanter als Perl, regelmäßiger als Python,
% flexibler als Ruby, typisierter als C#, robuster als Java und hat
% absolut nichts mit PHP gemeinsam? Es ist Haskell!

\begin{frame}[fragile]\frametitle{Quicksort in C}
  \begin{columns}
    \begin{column}[b]{0.6\textwidth}
      \scriptsize
      \begin{minted}{c}
// von rosettacode.org
void quick_sort (int *a, int n) {
    int i, j, p, t;
    if (n < 2)
        return;
    p = a[n / 2];
    for (i = 0, j = n - 1;; i++, j--) {
        while (a[i] < p)
            i++;
        while (p < a[j])
            j--;
        if (i >= j)
            break;
        t = a[i];
        a[i] = a[j];
        a[j] = t;
    }
    quick_sort(a, i);
    quick_sort(a + i, n - i);
}
      \end{minted}
    \end{column}

    \begin{column}{0.2\textwidth}
      % http://freephotos.atguru.in/hdphotos/sad-cat/sad-cat-13555.jpg
      \includegraphics[scale=0.3,flip]{images/sad-cat.jpg}
    \end{column}
  \end{columns}
\end{frame}

\begin{comment}
\begin{frame}[fragile]\frametitle{Quicksort in C\#}
  \tiny\vspace{-0.5em}
  \begin{minted}{csharp}
// von rosettacode.org
namespace Sort {
  using System;
 
  class QuickSort<T> where T : IComparable {
    #region Constants
    private const Int32 insertionLimitDefault = 16;
    private const Int32 pivotSamples = 5;
    #endregion
 
    #region Properties
    public Int32 InsertionLimit { get; set; }
    protected Random Random { get; set; }
    #endregion
 
    #region Constructors
    public QuickSort()
      : this(insertionLimitDefault, new Random()) {
    }
 
    public QuickSort(Int32 insertionLimit, Random random) {
      InsertionLimit = insertionLimit;
      Random = random;
    }
    #endregion
 
    #region Sort Methods
    public void Sort(T[] entries) {
      Sort(entries, 0, entries.Length - 1);
    }
 
    public void Sort(T[] entries, Int32 first, Int32 last) {
      var length = last + 1 - first;
      // Elide tail recursion by looping over the longer partition
      while (length > 1) {
        if (length < InsertionLimit) {
          InsertionSort<T>.Sort(entries, first, last);
          return;
        }
 
        var median = pivot(entries, first, last);
 
        var left = first;
        var right = last;
        partition(entries, median, ref left, ref right);
 
        var leftLength = right + 1 - first;
        var rightLength = last + 1 - left;
 
        if (leftLength < rightLength) {
          Sort(entries, first, right);
          first = left;
          length = rightLength;
        }
        else {
          Sort(entries, left, last);
          last = right;
          length = leftLength;
        }
      }
    }
 
    private T pivot(T[] entries, Int32 first, Int32 last) {
      var length = last + 1 - first;
      var sampleSize = Math.Min(pivotSamples, length);
      var right = first + sampleSize - 1;
      for (var left = first; left <= right; left++) {
        // Random sampling avoids pathological cases
        var random = Random.Next(left, last + 1);
        // Sample without replacement
        if (left != random)
          Swap(entries, left, random);
      }
 
      InsertionSort<T>.Sort(entries, first, right);
      return entries[first + sampleSize / 2];
    }
 
    private static void partition(T[] entries, T pivot, ref Int32 left, ref Int32 right) {
      while (left <= right) {
        while (pivot.CompareTo(entries[left]) > 0)
          left++;                       // pivot follows entry
        while (pivot.CompareTo(entries[right]) < 0)
          right--;                      // pivot precedes entry
 
        if (left < right)               // Move entries to their correct partition
          Swap(entries, left++, right--);
        else if (left == right) {       // No swap needed
          left++;
          right--;
        }
      }
    }
 
    public static void Swap(T[] entries, Int32 index1, Int32 index2) {
      var entry = entries[index1];
      entries[index1] = entries[index2];
      entries[index2] = entry;
    }
    #endregion
  }
 
  #region Insertion Sort
  static class InsertionSort<T> where T : IComparable {
    public static void Sort(T[] entries, Int32 first, Int32 last) {
      for (var i = first + 1; i <= last; i++) {
        var entry = entries[i];
        var j = i;
        while (j > first && entries[j - 1].CompareTo(entry) > 0)
          entries[j] = entries[--j];
        entries[j] = entry;
      }
    }
  }
  #endregion
}
  \end{minted}
\end{frame}
\end{comment}

\begin{frame}[fragile]\frametitle{Haskell, eine rein funktionale Sprache}
  \large\vspace{-0.5em}
  \begin{minted}{haskell}
qsort []     = []
qsort (x:xs) =
    qsort kleinere ++ [x] ++ qsort groessere
    where
    kleinere  = [y | y <- xs, y <= x]
    groessere = [y | y <- xs, y > x]
  \end{minted}
  \vfill

  \only<1>{
    \vspace{1em}
    \begin{center}
      \includegraphics[scale=0.35]{images/katze.jpg}
    \end{center}
  }

  \pause
  Die Fibonaccizahlen: 0, 1, 1, 2, 3, 5, 8, 13, 21, 34, \ldots
  \begin{minted}{haskell}
fibs = 0 : 1 : zipWith (+) fibs (tail fibs)
  \end{minted}

  \pause
  \begin{center}
  \hil{$\varheart$ Statisches Typsystem mit Typerschlie\ss ung $\varheart$} \\
  \hil{rein funktional} \textbullet{}
  \hil{nebenl\"aufig} \textbullet{}
  \hil{lazy} \textbullet{}
  \hil{7000\textsuperscript{+} Pakete}
  \end{center}
\end{frame}

% <audreyt>
% Perl: "Easy things are easy, hard things are possible"
% Haskell: "Hard things are easy, the impossible just happened"

\end{document}

Achtung. Bei der Nerdnacht hatte ich vergessen zu erklären, dass das
Quicksort-Beispiel Pattern Matching verwendet. Das hätte nicht passieren
dürfen! Beim Thema Typsystem hätte ich außerdem einen Kommentar der Art
"Tatsächlich ist das Typsystem so stark, dass es folgendes Motto gibt: Code,
der einmal kompiliert, ist bereits korrekt."

Diese Folien waren für fünf oder sechs Minuten konzipiert. Für einen
10-minütigen Lightning Talk könnte man das ganze besser machen:

* Live-Coding statt fertiger Code auf der Folie.

* Bessere Erklärung des Fibonacci-Beispiels, zum Beispiel so:

  fibs      = 0 :  1  : ???
  tail fibs = 1 : ???

  zipWith (+) fibs (tail fibs)
            = 1 : ???

  Danach aktualisiert man die Folie:

  fibs      = 0 :  1  :  1  : ???
  tail fibs = 1 :  1  : ???

  zipWith (+) fibs (tail fibs)
            = 1 : 2 : ???

  Und so weiter.

* Stichwörter, wo Haskell eingesetzt wird.
