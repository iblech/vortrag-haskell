\documentclass[12pt,compress,ngerman]{beamer}
\usepackage{amsmath}
\usepackage{url}
\usepackage{ucs}
\usepackage[utf8x]{inputenc}
\usepackage[ngerman]{babel}
\usepackage{ulem}  % sout
\usepackage{multicol}

% Manual syntax highlighting
\newcommand{\synfunc}   [1]{\color{blue!50!black}#1\color{black}}
\newcommand{\synstr}    [1]{\color{red!50!black}#1\color{black}}
\newcommand{\synvar}    [1]{\color{purple!50!black}#1\color{black}}
\newcommand{\synclass}  [1]{\color{green!50!black}#1\color{black}}
\newcommand{\syncomment}[1]{\color{blue!20!black}#1\color{black}}
\newcommand{\syncool}   [1]{\color{beamer@blendedblue}#1\color{black}}
\newcommand{\synoder}      {\ \ \color{black}$\vee$\ \ }
\newcommand{\hr}        {\rule[4pt]{\textwidth}{0.1pt}\\}
\newcommand{\hicolor}   [1]{\color[rgb]{0.6,0.2,0.8}#1\color{black}}
\newcommand{\synhilight}[1]{\hicolor{\textbf{#1}}}

\newcommand{\doofcomment}{\ \ \syncomment{\# :-(}}
\newcommand{\gutcomment} {\ \ \syncomment{\# :)}}

\newcommand{\T}[1]{\mathbf{#1}}
\newcommand{\Spur}[1]{\operatorname{Spur}{#1}}
\newcommand{\ul}[1]{\mathcal{#1}}
\newcommand{\singabb}[3]{$\begin{array}{@{}c@{}}\text{\includegraphics[scale=#3]{abb/#1.png}}\\\text{#2}\end{array}$}

\title{Haskell, \\eine rein funktionale Programmiersprache}
\author{Ingo Blechschmidt \\ \texttt{<iblech@web.de>}} % \\\texttt{<carina.willbold@student.uni-augsburg.de>}}
%\institute{{\footnotesize Universität Augsburg}}
\date{Augsburger Linux-Infotag 2010}

\usetheme{Warsaw}  %Warsaw, Berkeley?
\usecolortheme{seahorse}
\usefonttheme{serif}
\useinnertheme{rectangles}
\usepackage{bookman}
\setbeamercovered{transparent}

\setbeamertemplate{navigation symbols}{}
%\setbeamertemplate{headline}{}

\begin{document}

\frame{\titlepage}

\frame[plain]{\begin{center}
  \includegraphics[scale=0.35]{images/learn-you-a-haskell-for-great-good.png}
\end{center}}

\section{Grundlegendes}
\frame[t]{\frametitle{Haskell ist komisch!}
  \hfill\begin{picture}(0,0)(100,110)\includegraphics[scale=0.2]{images/endo.png}\end{picture}

  Haskell ist rein funktional:
  \begin{itemize}
    \item keine veränderliche Variablen
    \item keine Seiteneffekte
  \end{itemize}
}

\frame[t]{\frametitle{Bedarfsauswertung}
  \texttt{%
    \synvar{natürlicheZahlen}\ :: [\synclass{Integer}] \\
    \synvar{natürlicheZahlen}\ = [1..] \\
    \ \\
    \synvar{ungeradeZahlen}\ :: [\synclass{Integer}] \\
    \synvar{ungeradeZahlen}\ = \synfunc{filter}\ \synfunc{odd}\ [1..] \\
    \ \\
    \synvar{fibs}\ :: [\synclass{Integer}] \\
    \synvar{fibs}\ = 0 : 1 : \synfunc{zipWith}\ (+) \synvar{fibs}\ (\synfunc{tail}\ \synvar{fibs})
  }
}

\appendix
\section{Bildquellen}
\frame[t]{\frametitle{Bildquellen}
  \begin{itemize}
    \item \url{http://learnyouahaskell.com/splash.png}
    \item \url{http://save-endo.cs.uu.nl/target.png}
  \end{itemize}
}

\end{document}
